\documentclass{article}
\usepackage[utf8]{inputenc}
\usepackage{hyperref}
\usepackage{ragged2e}
\usepackage{amsmath}
\usepackage{amssymb}
\usepackage{amsfonts}
\usepackage{multicol}
\usepackage{tabularx}
\usepackage[a4paper, width=216mm, height=297mm, margin=3cm]{geometry}
\usepackage{gensymb}
\usepackage{enumerate}
\usepackage{setspace}
\usepackage{fancyhdr}
\usepackage{tikz}
\usepackage{array}
\usepackage{pgfplots}
\DeclareMathOperator{\sech}{sech}
\pgfplotsset{compat=1.15}
\usetikzlibrary{arrows}
\pagestyle{fancy}
\fancyhf{}
\rhead{Kuis 2 Matematika I Kelas 53}
\lhead{Halaman \thepage}
\hypersetup{
    colorlinks=true,
    linkcolor=blue,
    filecolor=blue,      
    urlcolor=blue,
}
\setlength{\columnsep}{0.8cm}
\begin{document}
 \begin{titlepage}
    \vspace*{\fill}
    \begin{center}
      \Huge {SOAL KUIS 2\\MATEMATIKA I \\KELAS 53}
    \end{center}
    \vspace*{\fill}
  \end{titlepage}
\makeatletter
\renewcommand*\env@matrix[1][*\c@MaxMatrixCols c]{%
  \hskip -\arraycolsep
  \let\@ifnextchar\new@ifnextchar
  \array{#1}}
\makeatother
\newcount\arrowcount
\newcommand\arrows[1]{
        \global\arrowcount#1
        \ifnum\arrowcount>0
                \begin{matrix}[c]
                \expandafter\nextarrow
        \fi
}

\newcommand\nextarrow[1]{
        \global\advance\arrowcount-1
        \ifx\relax#1\relax\else \xrightarrow{#1}\fi
        \ifnum\arrowcount=0
                \end{matrix}
        \else
                \\
                \expandafter\nextarrow
        \fi
}
\newcommand*\oline[1]{%
  \vbox{%
    \hrule height 0.5pt%                  % Line above with certain width
    \kern0.25ex%                          % Distance between line and content
    \hbox{%
      \kern-0.1em%                        % Distance between content and left side of box, negative values for lines shorter than content
      \ifmmode#1\else\ensuremath{#1}\fi%  % The content, typeset in dependence of mode
      \kern-0.1em%                        % Distance between content and left side of box, negative values for lines shorter than content
    }% end of hbox
  }% end of vbox
}
\newpage
\setstretch{1.3}
\section*{Soal Kuis 2 (Waktu 75 Menit)}
\begin{enumerate}
	\item Hitunglah $(-1+3i)^2 \left(\dfrac{2}{1-i}+\dfrac{2-i}{1+i}\right)$.
	\item Misalkan $z_1=1+i, z_2=3-2i$, hitunglah $\left|\dfrac{2z_1+\oline{z_2}-i}{z_2-2z_1+7}\right|$.
	\item Buktikan $\oline{z_1+z_2}=\oline{z_1}+\oline{z_2}$. \textit{(Petunjuk: misalkan $z_1=a+bi$ dan $z_2=c+di$)}
	\item Hitunglah $(2-2\sqrt{3}i)^{10}$.
	\item Dapatkan akar-akar kompleks dari persamaan $z^3-i=0$ dan gambarkan akar-akarnya pada bidang kompleks.
\end{enumerate}
\end{document}
