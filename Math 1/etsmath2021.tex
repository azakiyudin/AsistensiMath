\documentclass{article}
\usepackage[utf8]{inputenc}
\usepackage{hyperref,ragged2e,amsmath,multicol,setspace,
fancyhdr,amsfonts,tikz,pgfplots,nccmath,enumerate,verbatim}
\usepackage[a4paper, width=216mm, height=297mm, margin=3cm]{geometry}
\usepgfplotslibrary{polar,fillbetween}
\usepgflibrary{shapes.geometric}
\usetikzlibrary{calc,patterns,arrows}
\newcommand\mylog[1]{\mathop{{}^{#1}\mathrm{log}}}
\pgfplotsset{compat=1.15}
\pgfplotsset{my style/.append style={axis x line=middle, axis y line=
middle, xlabel={$x$}, ylabel={$y$}, axis equal }}
\usepackage{etoolbox}
\newcommand{\zerodisplayskips}{%
  \setlength{\abovedisplayskip}{0pt}%
  \setlength{\belowdisplayskip}{0pt}%
  \setlength{\abovedisplayshortskip}{0pt}%
  \setlength{\belowdisplayshortskip}{0pt}}
\pagestyle{fancy}
\fancyhf{}
\lhead{Halaman \thepage}
\rhead{Pembahasan Soal ETS 2021/2022 \\ (\href{https://instagram.com/ahmadzakiyudin_/}{@ahmadzakiyudin\_})}
\hypersetup{
    colorlinks=true,
    linkcolor=blue,
    filecolor=blue,      
    urlcolor=blue,
}
\setlength{\columnsep}{0.8cm}
\begin{document}
 \begin{titlepage}
    \vspace*{\fill}
    \begin{center}
      \Huge {PEMBAHASAN SOAL ETS \\ MATEMATIKA I \\ TAHUN 2021/2022}\\[0.4 cm]
      \huge {Ahmad Hisbu Zakiyudin}
    \end{center}
    \vspace*{\fill}
  \end{titlepage}
\makeatletter
\renewcommand*\env@matrix[1][*\c@MaxMatrixCols c]{%
  \hskip -\arraycolsep
  \let\@ifnextchar\new@ifnextchar
  \array{#1}}
\makeatother
\newcount\arrowcount
\newcommand\arrows[1]{
        \global\arrowcount#1
        \ifnum\arrowcount>0
                \begin{matrix}[c]
                \expandafter\nextarrow
        \fi
}
 
\newcommand\nextarrow[1]{
        \global\advance\arrowcount-1
        \ifx\relax#1\relax\else \xrightarrow{#1}\fi
        \ifnum\arrowcount=0
                \end{matrix}
        \else
                \\
                \expandafter\nextarrow
        \fi
}
\newpage
\setstretch{1.3}
\section*{Soal Kelas 1-10}
\begin{enumerate}
    \item Selesaikan pertidaksamaan berikut:
    \begin{align*}
    \dfrac{a}{x+1}-\dfrac{b}{x+2}\geq 0
    \end{align*}
    dimana $a,b$ adalah dua digit terakhir NRP.\\
    \textbf{Contoh:} Jika NRP anda adalah 5002201148, maka gunakan $a=4,b=8$; Jika $a$ atau $b$ adalah 0 ganti dengan angka 10.\\
    \textbf{Penyelesaian:}\\
    Tinjau syarat penyebut yaitu $x+1\neq 0$ dan $x+2\neq 0$ sehingga $x\neq -1$ dan $x\neq -2$. Selanjutnya perhatikan bahwa
    \begin{align*}
    \dfrac{a}{x+1}-\dfrac{b}{x+2} = \dfrac{a(x+2)-b(x+1)}{(x+1)(x+2)} = \dfrac{(a-b)x+2a-b}{(x+1)(x+2)} \geq 0 
    \end{align*}
    Jika $a=4$ dan $b=8$ diperoleh 
    \begin{equation}
    \dfrac{-4x}{(x+1)(x+2)}\geq 0
    \end{equation}
    sehingga pembuat nolnya adalah $x=0$ dan titik kritisnya $x=-2,x=-1$. \\
    Jika $x<-2$, maka persamaan (1) bernilai positif. \\
    Jika $-2<x<-1$, maka persamaan (1) bernilai negatif.\\
    Jika $-1<x\leq 0$, maka persamaan (1) bernilai positif atau 0.\\
    Jika $x>0$, maka persamaan (1) bernilai negatif.\\
    Penyelesaiannya adalah daerah yang bernilai tak negatif yaitu $x<-2$ atau $-1<x\leq 0$.\\
    Himpunan penyelesaiannya adalah $\{x\in \mathbb{R}~|~ x<-2 \vee -1<x\leq 0\}$.
    \item Hitunglah $(-i-1)^{49}(\cos \frac{\pi}{40}+i\sin\frac{\pi}{40})^{20}$\\
    \textbf{Penyelesaian:}\\
    Misalkan $z=-i-1$, maka $r=|z|=\sqrt{(-1)^2+(-1)^2}=\sqrt{2}$ dan $\tan \theta = \dfrac{-1}{-1}=1$ pada kuadran III sehingga $\theta=-\dfrac{3\pi}{4}$. Diperoleh $z=\sqrt{2}\left(\cos (-\frac{3\pi}{4})+i\sin(-\frac{3\pi}{4})\right)$. \\
    Ingat bahwa jika $z=r(\cos \theta +i\sin \theta)$ maka $z^n = r^n (\cos n\theta+i\sin n\theta)$ sehingga 
    \begin{align*}
    z^{49} &=\left(\sqrt{2}\right)^{49}\left[\cos \left(-\frac{3\pi}{4}\cdot 49\right)+i\sin\left(-\frac{3\pi}{4}\cdot 49\right)\right]\\
    &= \left(\sqrt{2}\right)^{49} \left[\cos \left(-\frac{147\pi}{4}+38\pi\right)+i\sin\left(-\frac{147\pi}{4}+38\pi\right)\right]\\
    &= \left(\sqrt{2}\right)^{49} \left[\cos \left(\frac{5\pi}{4}\right)+i\sin\left(\frac{5\pi}{4}\right)\right]\\
    &= \left(\sqrt{2}\right)^{49} \left[-\dfrac{1}{2}\sqrt{2}-\dfrac{i}{2}\sqrt{2}\right]\\
    &= -2^{24}(1+i)
    \end{align*}
    Dapat diperoleh pula untuk $w=\cos \frac{\pi}{40}+i\sin \frac{\pi}{40}$, maka 
    \begin{align*}
    w^{20} &= \cos \left(\frac{\pi}{40}\cdot 20\right)+i\sin \left(\frac{\pi}{40}\cdot 20\right) \\
    &= \cos \frac{\pi}{2}+i\sin \frac{\pi}{2}\\
    &= i
    \end{align*}
    Jadi 
    \begin{align*}
    (-i-1)^{49}\left(\cos \frac{\pi}{40}+i\sin\frac{\pi}{40}\right)^{20} = -2^{24}(1+i)i = -2^{24}(i-1) = 2^{24} (1-i)
    \end{align*}
    \item Gunakan aturan Cramer untuk menyelesaikan sistem persamaan linear berikut:
    \begin{align*}
    -\dfrac{2}{t}-\dfrac{1}{u}-\dfrac{3}{v} &= 3\\
    \dfrac{2}{t}-\dfrac{3}{u}+\dfrac{1}{v} &= -13\\
    \dfrac{2}{t}-\dfrac{3}{v} &= -11
    \end{align*}
    \textbf{Penyelesaian:}\\
    Misalkan $x=\dfrac{1}{t}, ~y=\dfrac{1}{u},$ dan $z=\dfrac{1}{v}$, maka diperoleh persamaan 
    \begin{align*}
    -2x-y-3z &= 3\\
    2x-3y+z &= -13\\
    2x-3z &= -11
    \end{align*}
    Dapat diperoleh matriks $A$ dan $b$ sebagai berikut
    \begin{align*}
    A = \begin{bmatrix}
    -2 & -1 & -3\\
    2 & -3 & 1\\
    2 & 0 & -3 
    \end{bmatrix} \qquad \text{dan} \qquad b = \begin{bmatrix}
    3 \\ -13 \\ -11
    \end{bmatrix}
    \end{align*}
    Selanjutnya, dapat dicari solusi $x,y,z$ dengan aturan cramer, yaitu 
    \begin{align*}
    x = \dfrac{\det A_1}{\det A} \qquad \qquad y=\dfrac{\det A_2}{\det A} \qquad \qquad z=\dfrac{\det A_3}{\det A}
    \end{align*}
    dengan $A_i$ merupakan matriks $A$ yang kolom ke-$i$ diganti dengan matriks $b$, sehingga 
    \begin{align*}
    A_1 = \begin{bmatrix}
    3 & -1 & -3\\
    -13 & -3 & 1\\
    -11 & 0 & -3 
    \end{bmatrix} \qquad A_2 = \begin{bmatrix}
    -2 & 3 & -3\\
    2 & -13 & 1\\
    2 & -11 & -3 
    \end{bmatrix} \qquad A_3 = \begin{bmatrix}
    -2 & -1 & 3\\
    2 & -3 & -13\\
    2 & 0 & -11 
    \end{bmatrix}
    \end{align*}
    Dengan menggunakan aturan Sarrus, dapat diperoleh 
    \begin{align*}
    \det A =~ &(-2)(-3)(-3)+(-1)(1)(2)+(-3)(2)(0)\\
    &-(-3)(-3)(2)-(-2)(1)(0)-(-1)(2)(-3) \\
    = ~& -44\\
    \det A_1 = ~ &(3)(-3)(-3)+(-1)(1)(-11)+(-3)(-13)(0)\\
    &-(-3)(-3)(-11)-(3)(1)(0)-(-1)(-13)(-3)\\
    = ~&176\\
    \det A_2 = ~ &(-2)(-13)(-3)+(3)(1)(2)+(-3)(2)(-11)\\
    &-(-3)(-13)(2)-(-2)(1)(-11)-(3)(2)(-3)\\
    = ~& -88 \\
    \det A_3 = ~&(-2)(-3)(-11)+(-1)(-13)(2)+(3)(2)(0)\\
    &-(3)(-3)(2)-(-2)(-13)(0)-(-1)(2)(-11) \\
    = ~&-44
    \end{align*}
    sehingga 
    \begin{align*}
    x = \dfrac{176}{-44} = -4 \qquad \qquad y = \dfrac{-88}{-44} = 2 \qquad \qquad z = \dfrac{-44}{-44} = 1
    \end{align*}
    Diperoleh $t=-\dfrac{1}{4},~ u=\dfrac{1}{2},$ dan $v=1$.
    \item Diberikan fungsi $f(x)=\sqrt{x^2-1}$ dan $g(x)=\dfrac{2}{x}$. Dapatkan 
    \begin{enumerate}
        \item $(f\circ g)(x)$ beserta domainnya
        \item $(g\circ f)(x)$ beserta domainnya
    \end{enumerate}
    \textbf{Penyelesaian:}
    \begin{enumerate}
        \item Perhatikan bahwa 
        \begin{align*}
        (f\circ g)(x) = f(g(x)) = \sqrt{\left(\dfrac{2}{x}\right)^2-1} = \sqrt{\dfrac{4-x^2}{x^2}} = \dfrac{\sqrt{4-x^2}}{|x|}
        \end{align*}
        Tinjau bahwa domain dari $(f\circ g) (x)$ adalah 
        $$ D_{f\circ g} = \{x\in D_g ~| ~g(x) \in D_f\}$$
        Kita punya $D_g = \{x\in \mathbb{R} ~|~ x\neq 0\} $ dan $D_f = \{x\in \mathbb{R}~ | ~x\leq -1 \vee x\geq 1\}$ sehingga $D_{f\circ g} = \left\{x\neq 0 ~\bigg|~ \dfrac{2}{x}\leq -1 \vee \dfrac{2}{x}\geq 1\right\}$. \\
        Untuk $\dfrac{2}{x}\leq-1$, diperoleh $\dfrac{2+x}{x}\leq 0$ sehingga $-2\leq x<0$.\\
        Untuk $\dfrac{2}{x}\geq 1$, diperoleh $\dfrac{2-x}{x}\geq 0$ sehingga $0<x\leq 2$.\\
        Jadi $D_{f\circ g} = \{x\in \mathbb{R} ~|~ -2\leq x<0 \vee 0<x\leq 2\}$.
        \item Perhatikan bahwa 
        \begin{align*}
        (g\circ f)(x) = g(f(x)) = \dfrac{2}{\sqrt{x^2-1}} = \dfrac{2\sqrt{x^2-1}}{x^2-1}
        \end{align*}
        Tinjau bahwa domain dari $(f\circ g) (x)$ adalah 
        $$ D_{g\circ f} = \{x\in D_f ~| ~f(x) \in D_g\}$$
        Kita punya $D_g = \{x\in \mathbb{R} ~|~ x\neq 0\} $ dan $D_f = \{x\in \mathbb{R}~ | ~x\leq -1 \vee x\geq 1\}$ sehingga $D_{g\circ f} = \{x\leq -1 \vee x\geq 1 ~|~ x\neq 0\}$ atau $D_{g\circ f} = \{x\in \mathbb{R} ~|~ x\leq -1 \vee x\geq 1\}$.
    \end{enumerate}
    \item Diberikan fungsi $f(x)=\dfrac{x^2-9}{|x|-3}$
    \begin{enumerate}
        \item Nyatakan $f(x)$ dalam bentuk fungsi sepotong-sepotong
        \item Selidiki di titik mana $f(x)$ diskontinu
    \end{enumerate}
    \textbf{Penyelesaian:}
    \begin{enumerate}
        \item Tinjau bahwa $|x| = \begin{cases} x, ~~&x\geq 0\\ -x, ~~&x<0\end{cases}$\\ sehingga untuk $x\geq 0$, maka $f(x)=\dfrac{x^2-9}{x-3} = \dfrac{(x-3)(x+3)}{x-3} = x+3$ dengan $x\neq 3$\\
        dan untuk $x<0$, maka $f(x)=\dfrac{x^2-9}{-x-3} = \dfrac{(x-3)(x+3)}{-(x+3)} = 3-x$ dengan $x\neq -3$\\
        Jadi diperoleh 
        \begin{align*}
        f(x) = \begin{cases} x+3, ~~ & x\geq 0 \wedge x\neq 3\\ 3-x, ~~ & x<0 \wedge x\neq -3 \end{cases}
        \end{align*}
        \item Tinjau bahwa $f(x)$ tidak terdefinisi pada titik $x=3$ dan $x=-3$ sehingga diskontinu pada titik tersebut. Akan tetapi, titik diskontinuitasnya dapat dihilangkan dengan mendefinisikan $f(x)$ pada titik $x=3$ dengan limitnya, yaitu $\displaystyle \lim_{x\rightarrow 3} f(x) = 6$, serta titik $x=-3$ dengan $\displaystyle \lim_{x\rightarrow -3} f(x) = 6$
    \end{enumerate}
\end{enumerate}
\end{document}
 