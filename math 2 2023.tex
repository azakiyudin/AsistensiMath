\documentclass{article}
\usepackage[utf8]{inputenc}
\usepackage{hyperref,ragged2e,amsmath,multicol,setspace,
fancyhdr,amsfonts,tikz,pgfplots,nccmath,enumerate,verbatim}
\usepackage[a4paper, width=216mm, height=297mm, margin=3cm]{geometry}
\usepgfplotslibrary{polar,fillbetween}
\usepgflibrary{shapes.geometric}
\usetikzlibrary{calc,patterns,arrows}
\newcommand\mylog[1]{\mathop{{}^{#1}\mathrm{log}}}
\pgfplotsset{compat=1.15}
\pgfplotsset{my style/.append style={axis x line=middle, axis y line=
middle, xlabel={$x$}, ylabel={$y$}, axis equal }}
\usepackage{etoolbox}
\newcommand{\zerodisplayskips}{%
  \setlength{\abovedisplayskip}{0pt}%
  \setlength{\belowdisplayskip}{0pt}%
  \setlength{\abovedisplayshortskip}{0pt}%
  \setlength{\belowdisplayshortskip}{0pt}}
\pagestyle{fancy}
\fancyhf{}
\lhead{Halaman \thepage}
\rhead{Rangkuman\\ (\href{https://instagram.com/ahmadzakiyudin_/}{@ahmadzakiyudin\_})}
\hypersetup{
    colorlinks=true,
    linkcolor=blue,
    filecolor=blue,      
    urlcolor=blue,
}
\setlength{\columnsep}{0.8cm}
\begin{document}
 \begin{titlepage}
    \vspace*{\fill}
    \begin{center}
      \Huge {RANGKUMAN \\ MATEMATIKA 2}\\[0.4 cm]
      \huge {Ahmad Hisbu Zakiyudin}
    \end{center}
    \vspace*{\fill}
  \end{titlepage}
\makeatletter
\renewcommand*\env@matrix[1][*\c@MaxMatrixCols c]{%
  \hskip -\arraycolsep
  \let\@ifnextchar\new@ifnextchar
  \array{#1}}
\makeatother
\newcount\arrowcount
\newcommand\arrows[1]{
        \global\arrowcount#1
        \ifnum\arrowcount>0
                \begin{matrix}[c]
                \expandafter\nextarrow
        \fi
}
 
\newcommand\nextarrow[1]{
        \global\advance\arrowcount-1
        \ifx\relax#1\relax\else \xrightarrow{#1}\fi
        \ifnum\arrowcount=0
                \end{matrix}
        \else
                \\
                \expandafter\nextarrow
        \fi
}
\newpage
\setstretch{1.3}
\section*{Sifat-Sifat Logaritma}
Untuk suatu $a,b,c>0$ dan $a\neq 1$ berlaku
\begin{multicols}{2}
\begin{enumerate}[a)]
	\item $\mylog{a} 1 = 0$
	\item $\mylog{a} a = 1$
	\item $\mylog{a} bc = \mylog{a} b + \mylog{a} c$
	\item $\mylog{a} \frac{b}{c} = \mylog{a} b - \mylog{a} c$
	\item $\mylog{a} b^r = r\mylog{a} b $
	\item $\mylog{a} \frac{1}{c} = -\mylog{a} c$
\end{enumerate}
\end{multicols}
\section*{Eksponensial Natural}
\begin{align*}
e=\lim_{x\rightarrow \infty} \left(1+\frac{1}{x}\right)^x \qquad\text{ dan } \qquad e = \lim_{x\rightarrow 0} (1+x)^{\frac{1}{x}}
\end{align*}
\section*{Teorema pada Turunan}
Jika $u=f(x)$ dan $v=g(x)$, maka 
\begin{enumerate}
	\item $\dfrac{d}{dx} [uv] = u'v + v'u$
	\item $\dfrac{d}{dx} \dfrac{u}{v} = \dfrac{u'v-v'u}{v^2}$ dengan $v\neq 0$
	\item $\dfrac{d}{dx} = \dfrac{d}{du}\dfrac{du}{dx}$\end{enumerate}
\section*{Logaritma Natural}
Untuk $x>0$ berlaku
\begin{enumerate}
	\item $\ln x = \displaystyle \int_1^x \dfrac{1}{t} \, dt$
	\item $\dfrac{d}{dx} \ln x = \dfrac{1}{x}$
\end{enumerate}
\section*{Turunan dan Integral Fungsi Eksponensial}
\begin{enumerate}
	\item $\dfrac{d}{dx} [e^x] = e^x$
	\item $\dfrac{d}{dx} [a^x] = a^x\ln a$
	\item $\displaystyle \int e^x \, dx = e^x +C$
	\item $\displaystyle \int a^x \, dx = \dfrac{a^x}{\ln a}$
\end{enumerate}
\section*{Turunan Fungsi Invers}
\begin{enumerate}
	\item $(f^{-1})'(x) = \dfrac{1}{f'(f^{-1}(x))}$
	\item $\dfrac{dy}{dx} = \dfrac{1}{dx/dy}$
\end{enumerate}
\section*{Turunan dan Integral Fungsi Invers Trigonometri}
\begin{multicols}{2}
\begin{enumerate}
	\item $\dfrac{d}{dx} [\sin^{-1}(x)] = \dfrac{1}{\sqrt{1-x^2}}$
	\item $\dfrac{d}{dx} [\cos^{-1}(x)] = -\dfrac{1}{\sqrt{1-x^2}}$
	\item $\dfrac{d}{dx} [\tan^{-1}(x)] = \dfrac{1}{1+x^2}$
	\item $\dfrac{d}{dx} [\cot^{-1}(x)] = -\dfrac{1}{1+x^2}$
	\item $\dfrac{d}{dx} [\sec^{-1}(x)] = \dfrac{1}{x\sqrt{x^2-1}}$
	\item $\dfrac{d}{dx} [\csc^{-1}(x)] = -\dfrac{1}{x\sqrt{x^2-1}}$
	\item $\displaystyle \int \dfrac{1}{\sqrt{1-x^2}}\, dx = [\sin^{-1}(x)]+ C$  
	\item $\displaystyle \int \dfrac{1}{1+x^2}\, dx = [\tan^{-1}(x)]+ C$  
	\item $\displaystyle \int \dfrac{1}{x\sqrt{x^2-1}}\, dx = [\sec^{-1}(x)]+ C$  
\end{enumerate}
\end{multicols}
\end{document}
