\documentclass{article}
\usepackage[utf8]{inputenc}
\usepackage{hyperref}
\usepackage{ragged2e}
\usepackage{amsmath}
\usepackage{amsfonts}
\usepackage{multicol}
\usepackage[a4paper, width=216mm, height=297mm, margin=3cm]{geometry}
\usepackage{gensymb}
\usepackage{setspace}
\usepackage{fancyhdr}
\usepackage{tikz}
\usepackage{pgfplots}
\DeclareMathOperator{\sech}{sech}
\pgfplotsset{compat=1.15}
\usetikzlibrary{arrows}
\pagestyle{fancy}
\fancyhf{}
\lhead{Halaman \thepage}
\rhead{Ahmad Hisbu Zakiyudin/5002201148}
\hypersetup{
    colorlinks=true,
    linkcolor=blue,
    filecolor=blue,      
    urlcolor=blue,
}
\setlength{\columnsep}{0.8cm}
\begin{document}
 \begin{titlepage}
    \vspace*{\fill}
    \begin{center}
      \Huge {Tugas 5 \\ Matematika 2 Kelas 67}\\[0.4cm] 
      \huge {Ahmad Hisbu Zakiyudin \\ 5002201148 \\ Departemen Matematika}\\[0.4cm]
    \end{center}
    \vspace*{\fill}
  \end{titlepage}
\newpage
\setstretch{1.3}
\section*{Latihan 2.1}
\begin{enumerate}
	\item[1h)] Selesaikan \( \displaystyle \int t^2\sin t \, dt \) \\
	Penyelesaian: \\ 
	Misalkan \[u=t^2 \qquad \text{dan} \qquad dv=\sin t \, dt\]
	maka \[ \frac{du}{dt} = 2t \qquad \text{atau} \qquad du = 2t \, dt\]
	dan \[ v = \int \sin t \, dt = -\cos t\]
	sehingga
	\begin{align*}
	\int t^2\sin t \, dt &= \int u \, dv \\
	&= uv - \int v \, du \\
	&= -t^2\cos t + 2\int t\cos t \, dt \tag{1.1}
	\end{align*}
	Kemudian kita hitung \( \displaystyle \int t\cos t \, dt \) \\
	Misalkan \[ u=t \qquad \text{dan} \qquad dv=\cos t \, dt\]
	maka \[ \frac{du}{dt} = 1 \qquad \text{atau} \qquad du=dt\]
	dan \[ v = \int \cos t \, dt = \sin t\]
	sehingga
	\begin{align*}
	\int t\cos t \, dt &= \int u \, dv \\
	&= uv - \int v \, du \\
	&= t\sin t - \int \sin t \, dt \\
	&= t\sin t + \cos t + C_1
	\end{align*}
	Substitusikan ke 1.1, diperoleh 
	\begin{align*}
	\int t^2\sin t \, dt &= -t^2\cos t + 2(t\sin t + \cos t + C_1) \\
	&= -t^2\cos t + 2t\sin t + 2\cos t + C
	\end{align*}
\newpage
	\item[13m)] Selesaikan \(\displaystyle \int \cos^4 2x\sin^3 2x \, dx\) \\
	Penyelesaian: 
	\[\int \cos^4 2x\sin^3 2x \, dx = \int \cos^4 2x(1-\cos^2 2x)\sin 2x \, dx\]
	Misalkan $u=\cos 2x$, sehingga $\dfrac{du}{dx} = -2\sin 2x$ atau $-\dfrac{du}{2}=\sin 2x \, dx$, diperoleh
	\begin{align*}
	\int \cos^4 2x\sin^3 2x \, dx &= -\frac{1}{2} \int u^4(1-u^2) \, du \\
	&= -\frac{1}{2} \int u^4 - u^6 \, du \\
	&= -\frac{1}{2} \bigg[\frac{u^5}{5} - \frac{u^7}{7}\bigg] + C \\
	&= \frac{u^7}{14} - \frac{u^5}{10} + C \\
	&= \frac{\cos^7 2x}{14} - \frac{\cos^5 2x}{10} + C
	\end{align*}
	\item[14p)] Selesaikan \(\displaystyle \int \tan\theta\sec^5\theta \, d\theta\) \\
	Penyelesaian: \\
	Misalkan $\sec\theta = u$, sehingga $\dfrac{du}{d\theta} = \tan\theta\sec\theta \, d\theta$, diperoleh
	\begin{align*}
	\int \tan\theta\sec^5\theta \, d\theta &= \int u^4 \, du \\
	&= \frac{u^5}{5} + C \\
	&= \frac{1}{5}\sec^5\theta + C
	\end{align*}
\end{enumerate}
\newpage
\section*{Latihan 2.2}
\begin{enumerate}
	\item[2k)] Selesaikan \(\displaystyle \int \frac{x^5+2x^^2+1}{x^3+x} \, dx\) \\
	Penyelesaian: \\
	Perhatikan bahwa
	\begin{align*}
	\frac{x^5+2x^2+1}{x^3+x} &= \frac{x^5+x^3-x^3-x+2x^2+x+1}{x^3+x} \\
	&= \frac{x^2(x^3+x)-(x^3+x)+2x^2+x+1}{x^3+x} \\
	&= x^2-1 + \frac{2x^2+x+1}{x(x^2+1)}
	\end{align*} 
	Dengan aturan faktor linear dan faktor kuadratik, diperoleh dekomposisi pecahannya adalah \[ \frac{2x^2+x+1}{x(x^2+1)} = \frac{A}{x} + \frac{Bx+C}{x^2+1} \]
	Kemudian 
	\begin{align*}
	2x^2+x+1 &\equiv A(x^2+1)+x(Bx+C) \\
	&\equiv Ax^2+A+Bx^2+Cx \\
	&\equiv (A+B)x^2+Cx+A
	\end{align*}
	Kita peroleh $A=1, B=1$ dan $C=1$, sehingga
	\begin{align*}
	\int \frac{x^5+2x^2+1}{x^3+x} \, dx &= \int \bigg[x^2-1+\frac{1}{x} + \frac{x}{x^2+1} + \frac{1}{x^2+1}\bigg] \, dx \\
	&= \frac{x^3}{3}-x+\ln x + \frac{1}{2}\ln(x^2+1) + \tan^{-1}x + C
	\end{align*}
	\item[6)] Selesaikan \(\displaystyle \int \frac{1}{16x^3-4x^2+4x-1} \, dx\) \\
	Penyelesaian: \\
	Perhatikan bahwa 
	\begin{align*}
	\frac{1}{16x^3-4x^2+4x-1} &= \frac{1}{4x^2(4x-1)+(4x-1)} \\
	&= \frac{1}{(4x^2+1)(4x-1)}
	\end{align*}
	Misalkan $u=2x$, sehingga $\dfrac{du}{dx}=2$ atau $\dfrac{du}{2}=dx$, maka \[\int \frac{1}{16x^3-4x^2+4x-1} \, dx = \frac{1}{2} \int \frac{1}{(u^2+1)(2u-1)} \, du\]	Dengan aturan faktor linear dan faktor kuadratik, diperoleh dekomposisi pecahannya adalah \[ \frac{1}{(u^2+1)(2u-1)} = \frac{A}{2u-1} + \frac{Bu+C}{u^2+1}\]
	Kemudian 
	\begin{align*}
	1 &\equiv A(u^2+1)+(2u-1)(Bu+C) \\
	&\equiv Au^2+A+2Bu^2+2Cu-Bu-C \\
	&\equiv (A+2B)u^2+(2C-B)u+A-C
	\end{align*}
	Kita peroleh $A+2B=0$, $2C-B=0$, dan $A-C=1$. Kemudian diperoleh bahwa $A=\dfrac{4}{5}$, $B=-\dfrac{2}{5}$, $C=-\dfrac{1}{5}$, sehingga
	\begin{align*}
	\int \frac{1}{16x^3-4x^2+4x-1} \, dx &= \frac{1}{2} \int \frac{1}{(u^2+1)(2u-1)} \, du \\
	&= \frac{1}{2} \int \bigg[\frac{4}{5(2u-1)} + \frac{-2u-1}{5(u^2+1)}\bigg] \, du \\
	&= \frac{2}{5} \int \frac{1}{2u-1} \, du - \frac{1}{5}\int \frac{u}{u^2+1} \, du - \frac{1}{10} \int \frac{1}{u^2+1} \, du   \\
	&= \frac{1}{5} \ln|2u-1| -\frac{1}{10}\ln|u^2+1|-\frac{1}{10}\tan^{-1}u + C \\
	&= \frac{1}{5} \ln|4x-1| -\frac{1}{10}\ln|4x^2+1|-\frac{1}{10}\tan^{-1}2x + C \\
	\end{align*}
\end{enumerate}
\newpage
\section*{Latihan 2.3}
\begin{enumerate}
	\item[1h)] Selesaikan \(\displaystyle \int \frac{1}{t^2\sqrt{t^2-9}} \, dt\) \\
	Penyelesaian: \\
	Misalkan $t=3\sec u$, sehingga $\dfrac{dt}{du}=3\sec u\tan u$ atau $dt = 3\sec u\tan u \,du$, maka
	\begin{align*}
	\int \frac{1}{t^2\sqrt{t^2-9}} \, dt\ &= \int \frac{3\sec u\tan u}{9\sec^2u\sqrt{9(\sec^2 u -1)}} \, du \\
	&= \frac{1}{3}\int \frac{\tan u}{\sec u\sqrt{9\tan^2 u}} \, du \\
	&= \frac{1}{9} \int cos u \, du \\
	&= \frac{1}{9} \sin u + C
	\end{align*}
	Kemudian, kita kembalikan $u$ ke $t$. Perhatikan bahwa $\cos u=\dfrac{3}{t}$, maka
	\begin{align*}
	\sin u &= \sqrt{1-\cos^2u} \\
	&= \sqrt{1-\frac{9}{t^2}} \\
	&= \frac{1}{t}\sqrt{t^2-9}
	\end{align*}
	Jadi \[\int \frac{1}{t^2\sqrt{t^2-9}} \, dt = \frac{\sqrt{t^2-9}}{9t} + C\]
	\item[5aa)] Selesaikan \(\displaystyle \int \frac{t^{2/3}}{t+1} \, dt\) \\
	Penyelesaian: \\
	Misalkan $u=t^{1/3}$, sehingga $u^2=t^{2/3}$ dan $u^3=t$. \\
	Kemudian \[\dfrac{du}{dt} = \dfrac{1}{3t^{2/3}} \qquad \text{atau} \qquad dt=3t^{2/3} \, du \qquad \text{atau} \qquad dt=3u^2 \, du \]
	Kita peroleh 
	\begin{align*}
	\int \frac{t^{2/3}}{t+1} \, dt &= \int \frac{u^4}{u^3+1} \, du \\
	&= \int \bigg[\frac{u(u^3+1)}{u^3+1} - \frac{u}{u^3+1}\bigg] \, du
	\end{align*}
	Kita selesaikan dulu \(\displaystyle \int \frac{u}{u^3+1} \, du\). Perhatikan bahwa $u^3+1=(u+1)(u^2-u+1)$ sehingga dekomposisinya adalah \[\frac{u}{(u+1)(u^2-u+1)} = \frac{A}{u+1} + \frac{Bu+C}{u^2-u+1}\]
	Kemudian 
	\begin{align*}
	u &\equiv A(u^2-u+1)+(u+1)(Bu+C) \\
	&\equiv Au^2-Au+A+Bu^2+Cu+Bu+C \\
	&\equiv (A+B)u^2+(B+C-A)u+A+C
	\end{align*}
	Kita peroleh $A+B=0$, $B+C-A=1$, dan $A+C=0$, sehingga $(A,B,C)=\Big(-\dfrac{1}{3},\dfrac{1}{3},\dfrac{1}{3}\Big)$. \\
	Jadi 
	\begin{align*}
	\int \frac{u}{u^3+1} \, du &= \int \bigg[-\frac{1}{3(u+1)}+\frac{u+1}{3(u^2-u+1)}\bigg] \, du \\
	&= \frac{1}{3} \int \frac{u+1}{u^2-u+1} \, du -\frac{1}{3} \int \frac{1}{u+1} \, du \\
	&= \frac{1}{6}\int \frac{2u-1}{u^2-u+1} \, du + \frac{1}{2} \int\frac{1}{u^2-u+1} \, du -\frac{1}{3} \int \frac{1}{u+1} \, du
	\end{align*}
	Kemudian kita selesaikan \(\displaystyle \int \frac{1}{u^2-u+1} \, du\). Perhatikan bahwa $u^2-u+1=\Big(u-\dfrac{1}{2}\Big)^2+\dfrac{3}{4}$. Misalkan $u-\dfrac{1}{2}=p$ sehingga $du=dp$, maka
	\begin{align*}
	\int \frac{1}{u^2-u+1} \, du &= \int \frac{1}{p^2+\frac{3}{4}} \, dp \\
	&= 4\int \frac{1}{4p^2+3} \, dp
	\end{align*}
	Misalkan $p=\dfrac{q\sqrt{3}}{2}$, sehingga $dp=\dfrac{\sqrt{3}}{2} \, dq$, maka
	\begin{align*}
	4\int \frac{1}{4p^2+3} \, dp &= 2\sqrt{3}\int \frac{1}{3q^2+3} \, dq \\
	&= \frac{2\sqrt{3}}{3}\int \frac{1}{q^2+1} \, dq \\
	&= \frac{2\sqrt{3}}{3} \tan^{-1}q \\
	&= \frac{2\sqrt{3}}{3} \tan^{-1}\Big(\frac{2p}{\sqrt{3}}\Big)
	\end{align*}
	Jadi 
	\begin{align*}
	\int \frac{1}{u^2-u+1} \, du &= 4\int \frac{1}{4p^2+3} \, dp \\
	&= \frac{2\sqrt{3}}{3} \tan^{-1}\Big(\frac{2p}{\sqrt{3}}\Big) \\
	&= \frac{2\sqrt{3}}{3} \tan^{-1}\Big(\frac{2u-1}{\sqrt{3}}\Big)
	\end{align*}
	Kemudian
	\begin{align*}
	\int \frac{u}{u^3+1} \, du &= \frac{1}{6}\int \frac{2u-1}{u^2-u+1} \, du + \frac{1}{2} \int\frac{1}{u^2-u+1} \, du -\frac{1}{3} \int \frac{1}{u+1} \, du \\
	&= \frac{1}{6}\ln|u^2-u+1|+\frac{2\sqrt{3}}{3} \tan^{-1}\Big(\frac{2u-1}{\sqrt{3}}\Big) - \frac{1}{3}\ln|u+1|
	\end{align*}
	Sehingga
	\begin{align*}
	\int \frac{t^{2/3}}{t+1} \, dt &= \int \frac{u^4}{u^3+1} \, du \\
	&= \int u \, du - \int \frac{u}{u^3+1} \, du \\
	&= \frac{u^2}{2} - \bigg[\frac{1}{6}\ln|u^2-u+1|+\frac{2\sqrt{3}}{3} \tan^{-1}\Big(\frac{2u-1}{\sqrt{3}}\Big) - \frac{1}{3}\ln|u+1|\bigg] + C\\
	&= \frac{u^2}{2} +\frac{1}{3}\ln|u+1|-\frac{1}{6}\ln|u^2-u+1|-\frac{2\sqrt{3}}{3} \tan^{-1}\Big(\frac{2u-1}{\sqrt{3}}\Big) + C\\
	&= \frac{\sqrt[3]{t^2}}{2} +\frac{1}{3}\ln|\sqrt[3]{t}+1|-\frac{1}{6}\ln|\sqrt[3]{t^2}-\sqrt[3]{t}+1|-\frac{2\sqrt{3}}{3} \tan^{-1}\Big(\frac{2\sqrt[3]{t}-1}{\sqrt{3}}\Big) + C
	\end{align*}
	\item[5jj)] Selesaikan \(\displaystyle \int \frac{1}{1+\sin x+\cos x} \, dx\)\\
	Penyelesaian: \\
	Misalkan $u=\tan\Big(\dfrac{x}{2}\Big)$, sehingga $x=2\tan^{-1}u$, dan $dx=\dfrac{2}{1+u^2} \, du$. \\
	Ingat bahwa $\tan\Big(\dfrac{x}{2}\Big)=\sqrt{\dfrac{1-\cos x}{1+\cos x}}$, sehingga 
	\begin{align*}
	u^2 &= \frac{1-\cos x}{1+\cos x} \\
	u^2+u^2\cos x &= 1-\cos x \\
	(u^2+1)\cos x &= 1-u^2 \\
	\cos x &= \frac{1-u^2}{1+u^2}
	\end{align*}
	Kemudian
	\begin{align*}
	\sin x &= \sqrt{1-\bigg(\frac{1-u^2}{1+u^2}\bigg)^2} \\
	&= \sqrt{\frac{(1+u^2)^2-(1-u^2)^2}{(1+u^2)^2}} \\
	&= \sqrt{\frac{(1+u^2+1-u^2)(1+u^2-1+u^2)}{(1+u^2)^2}} \\
	&= \sqrt{\frac{4u^2}{{(1+u^2)^2}}} \\
	&= \frac{2u}{1+u^2}
	\end{align*}
	Kita peroleh 
	\begin{align*}
	\int \frac{1}{1+\sin x+\cos x} \, dx &= \int \frac{1}{1+\bigg(\dfrac{2u}{1+u^2}\bigg)+\bigg(\dfrac{1-u^2}{1+u^2}\bigg)} \bigg(\frac{2}{1+u^2}\bigg) \, du \\
	&= \int \frac{2}{(1+u^2)+2u+(1-u^2)} \, du \\
	&= \int \frac{1}{1+u} \, du \\
	&= \ln|1+u| + C \\
	&= \ln\Big|1+\tan\Big(\frac{x}{2}\Big)\Big| + C
	\end{align*}
\end{enumerate}
\end{document}
