\documentclass{article}
\usepackage[utf8]{inputenc}
\usepackage{hyperref}
\usepackage{ragged2e}
\usepackage{amsmath}
\usepackage{amsfonts}
\usepackage{multicol}
\usepackage[a4paper, width=216mm, height=297mm, margin=3cm]{geometry}
\usepackage{gensymb}
\usepackage{setspace}
\usepackage{fancyhdr}
\usepackage{tikz}
\usepackage{pgfplots}
\DeclareMathOperator{\sech}{sech}
\pgfplotsset{compat=1.15}
\usetikzlibrary{arrows}
\pagestyle{fancy}
\fancyhf{}
\lhead{Halaman \thepage}
\rhead{Pembahasan ETS ALE}
\hypersetup{
    colorlinks=true,
    linkcolor=blue,
    filecolor=blue,      
    urlcolor=blue,
}
\setlength{\columnsep}{0.8cm}
\begin{document}
 \begin{titlepage}
    \vspace*{\fill}
    \begin{center}
      \Huge {Pembahasan ETS \\ Aljabar Linear Elementer}\\[0.4cm] 
      \huge {Ahmad Hisbu Zakiyudin \\ 5002201148 \\ Departemen Matematika}\\[0.4cm]
    \end{center}
    \vspace*{\fill}
  \end{titlepage}
\newpage
\setstretch{1.3}
\begin{enumerate}
	\item Syarat-syarat apakah yang harus dipenuhi $b_1,b_2,$ dan $b_3$ agar sistem persamaan di bawah ini konsisten
	\begin{align*}
	x_1+2x_2+3x_3 &= b_1 \\
	2x_1+5x_2+3x_3 &= b_2 \\
	x_1 \, \, \, \, \quad \quad + 8x_3 &= b_3
	\end{align*}
	\item Diketahui $\vec v_1 = \Big(\dfrac{2}{3},\dfrac{1}{3},\dfrac{2}{3}\Big)$ dan $\vec v_2 = \Big(\dfrac{1}{3},\dfrac{2}{3},-\dfrac{2}{3}\Big)$ adalah vektor-vektor yang ortonormal. Carilah sebuah vektor $\vec v_3$, sehingga $\{\vec v_1,\vec v_2, \vec v_3\}$ adalah himpunan ortonormal
	\item \begin{enumerate}
	\item[a.] Jika $M_{22} = himpunan \, semua \, matriks \, bujur\, sangkar\, order\, 2.$, maka tunjukkan bahwa $S=\Bigg\{\begin{bmatrix}
	3 & 6\\
	3 & -6
	\end{bmatrix},\begin{bmatrix}
	0 & -1\\
	-1 & 0
	\end{bmatrix},\begin{bmatrix}
	0 & -8\\
	-12 & -4
	\end{bmatrix},\begin{bmatrix}
	1 & 0\\
	-1 & 2
	\end{bmatrix}\Bigg\}$ adalah basis untuk $M_{22}$
	\item[b.] Tentukan vektor koordinat $\vec w$ relatif terhadap $S=\{\vec v_1,\vec v_2,\vec v_3\}$ dengan $\vec w = (5,-12,3)$; $\vec v_1 = (1,2,3)$; $\vec v_2 = (-4,5,6)$; $\vec v_3 = (7,-8,9)$
	\end{enumerate}
	\item Diketahui $S=\{(2,1,0),(1,-1,2),(0,3,-4)\}$ dan $V=span(S)$. Jika $(a,b,c) \in \mathbb{R}^3$, tentukan syarat/kondisi agar $(a,b,c) \in V$.
	\item Dapatkan suatu basis untuk ruang kosong dari matriks
\end{enumerate}
\end{document}
